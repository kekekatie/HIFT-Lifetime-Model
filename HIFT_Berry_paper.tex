\documentclass[11pt]{article}
\usepackage{amsmath,amssymb,amsfonts,graphicx,hyperref}
\usepackage{geometry}
\geometry{margin=1in}
\title{The Berry Curvature Origin of the Golden Ratio in a Projected $E_8$ Vacuum}
\author{HIFT Collaboration}
\date{\today}

\begin{document}
\maketitle

\begin{abstract}
We present a geometric and topological mechanism by which the golden ratio (and in particular a $\phi^2$ scale) may appear as a fundamental coupling constant of low-energy effective physics when a periodic, highly symmetric lattice in eight dimensions (the $E_8$ root lattice) is projected to three physical dimensions via a two-stage cut-and-project scheme.
\end{abstract}

\section{Introduction}
The exceptional $E_8$ lattice is a unique, maximally symmetric, and optimally packed structure in eight Euclidean dimensions. Projections of $E_8$ generate quasicrystalline geometries in lower dimensions.

\section{Projection, kernels and the Hilbert bundle}
\subsection{Two-pass projection}
We implement the projection as a composition:
\begin{equation}
\Pi = \pi_3 \circ F_{H4} \circ P_{E8}
\end{equation}

\section{Interface physics and fractionalization}
When domains with different integrated curvature indices meet, standard bulk-boundary correspondences predict interface modes.

\section{Algorithmic blueprint: the HIFT-Engine}
\textbf{High-level plan:}
\begin{enumerate}
\item Generate an $E_8$ point cloud truncated at radius $R$.
\item Choose projection matrices.
\item Cut-and-project: compute physical and perpendicular components.
\item Build kernel Hilbert bases.
\item Compute Berry connection and curvature.
\item Identify domains and locate seams.
\item Detect localized modes and measure fractionalization.
\end{enumerate}

\section{Conclusions}
This paper frames a testable program: the HIFT-Engine (numerical) followed by experimental tests. The central thesis—that the golden ratio can appear as a geometric quantization unit via Berry curvature arising from an $E_8$ projection—is falsifiable and numerically accessible.

\end{document}
